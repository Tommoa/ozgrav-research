\documentclass{article}
\usepackage[utf8]{inputenc}
\usepackage[a4paper]{geometry}
\usepackage[myheadings]{fullpage}
\usepackage[english]{babel}
\usepackage[T1]{fontenc}
\usepackage{sectsty}
\usepackage{appendix}
\usepackage{fancyhdr}
\usepackage{lastpage}
\usepackage{setspace}
\usepackage{amsmath}

\onehalfspacing

\pagestyle{fancy}
\fancyhf{}
\setlength\headheight{15pt}
\fancyhead[L]{Student ID: 21963144}
\fancyhead[R]{UWA}
\fancyfoot[R]{Page \thepage{} of \pageref{LastPage}}
\sectionfont{\scshape}

\newcommand{\HRule}[1]{\rule{\linewidth}{#1}}

\title{\normalsize \textsc{GENG5551 Research Project}
        \\ [1.5cm]
        \HRule{0.5pt} \\
        \LARGE \textbf{\uppercase{Research Proposal}}
        \HRule{2pt} \\ [0.5cm]
        \normalsize \today \vspace*{3\baselineskip}}

\author{Thomas Hill Almeida (21963144)}
\date{}

\begin{document}
\maketitle{}
\tableofcontents{}
\newpage{}

\section*{Introduction}
\section{Background}

\subsection{A brief history of gravitational waves}

In 1687, Isaac Newton published his book, ``\textit{Philosophi\ae{} Naturalis Principia Mathematica}'', containing his law of universal gravitation. It has since been taught in high-schools across the world in introductory physics classes with the formula \( F = G\dfrac{m_1m_2}{r^2} \), due in part to its simplicity and general accuracy for predicting movement. Despite its general applicability, one major issue of the theory is that it infers that gravitational is instantaneously applied, without any apparent method through which it could be transmitted.

Roughly two centuries later, in 1905, Albert Einstein presented the theory of special relativity in his paper ``\textit{Zur Elektrodynamik bewegter K{\"o}rper}'' (English: \textit{``On the Electrodynamics of Moving Bodies''}). The theory introduced the concept of \textbf{spacetime} to describe inertial reference frames as a four-dimensional coordinate system, \((t, x, y, z)\), where $t$ is time and $(x, y, z)$ are the three spatial dimensions. He further stated two important axioms; that the speed of light in a vacuum is the same for all observers regardless of motion and that the laws of physics are invariant in all inertial frames of reference. About ten years later, Albert Einstein incorporated the effect of gravity with special relativity, forming the general theory of relativity.

The general theory of relativity postulates that the effect of gravity can be characterised as each gravitational potential source changing the curvature of spacetime. The relationship of gravitational mass-energy and the shape of spacetime is given by Einstein's field equations:

\begin{equation}
    G_{\mu{}v} + \Lambda{}g_{\mu{}v} = \dfrac{8\pi{}G}{c^4}T_{\mu{}v}
\end{equation}

where \(T_{\mu{}v}\) is the stress-energy tensor \footnote{Tensors are very similar to matrices of vectors and are typically used to describe mathematical geometric relationships}, \(G_{\mu{}v}\) is the Einstein tensor, \(g_{\mu{}v}\) is the spacetime metric, \(\Lambda\) is the cosmological constant, \(G\) is the universal gravitational constant and \(c\) is the speed of light.

An implication of gravity curving spacetime is that massive accelerating objects would cause `ripples' in fabric of spacetime called gravitational waves. The existence of gravitational waves remained a theory until 1974, when Russell Hulse and Joseph Taylor discovered a binary pair of neutron stars that were orbiting each other. After several years of measurement, they found that the speed at which the stars were orbiting each other was slowing in a manner consistent with the predictions of the general theory of relativity, showing that gravitational waves did indeed exist.

\subsection{Gravitational wave detection}

Inteferometry is the use of wave interference \textemdash{} in which waves superpose on each other to create a resultant wave \textemdash{} to study various phenomena.

\section{Problem Description \& Goal}
\section{Methods}
\section{Proposed timeline}
\end{document}
