\documentclass{article}
\usepackage[utf8]{inputenc}
\usepackage[a4paper]{geometry}
\usepackage[myheadings]{fullpage}
\usepackage[english]{babel}
\usepackage[T1]{fontenc}
\usepackage{sectsty}
\usepackage{appendix}
\usepackage{fancyhdr}
\usepackage{lastpage}
\usepackage{setspace}
\usepackage{amsmath}
\usepackage[yyyymmdd]{datetime}
\usepackage{longtable}
\usepackage[table, xcdraw]{xcolor}
\usepackage[
    backend=bibtex,
    style=numeric,
    sorting=ynt
]{biblatex}
\addbibresource{bibliography.bib}

\renewcommand{\dateseparator}{-}

\onehalfspacing

\pagestyle{fancy}
\fancyhf{}
\setlength\headheight{15pt}
\fancyhead[L]{Student ID: 21963144}
\fancyhead[R]{UWA}
\fancyfoot[R]{Page \thepage{} of \pageref{LastPage}}
\sectionfont{\scshape}

\newcommand{\HRule}[1]{\rule{\linewidth}{#1}}

\title{\normalsize \textsc{GENG5551 Research Project}
        \\ [1.5cm]
        \HRule{0.5pt} \\
        \LARGE \textbf{\uppercase{Research Proposal}}
        \HRule{2pt} \\ [0.5cm]
        \normalsize \today \vspace*{3\baselineskip}}

\author{Thomas Hill Almeida (21963144)\\
\\
Supervisor:\\
Professor Linqing Wen
}
\date{}

\begin{document}
\maketitle{}
\tableofcontents{}
\newpage{}

\section{Introduction}
\subsection{Background}

In 1687, Isaac Newton published his book, ``\textit{Philosophi\ae{} Naturalis Principia
Mathematica}'', containing his law of universal gravitation:

\begin{equation}
    F = G\dfrac{m_1m_2}{r^2}
\end{equation}

where \(F\) is the resultant gravitational force, \(G\) is the universal gravitational constant,
\(m\) is the mass of the objects and \(r\) is the distance between the objects.

It has since been taught in high-schools across the world in introductory physics classes, due in
part to its simplicity and general accuracy for predicting movement. Despite its general
applicability, one major issue of the theory is that it infers that gravitational is instantaneously
applied, without any apparent method through which it could be transmitted.

Roughly two centuries later, in 1905, Albert Einstein presented the theory of special relativity
in his paper ``\textit{Zur Elektrodynamik bewegter K{\"o}rper}'' (English: \textit{``On the
Electrodynamics of Moving Bodies''}). The theory introduced the concept of \textbf{spacetime} to
describe inertial reference frames as a four-dimensional coordinate system, \((t, x, y, z)\), where
$t$ is time and \((x, y, z)\) are the three spatial dimensions. He further stated two important
axioms; that the speed of light in a vacuum is the same for all observers regardless of motion and
that the laws of physics are invariant in all inertial frames of reference. About ten years later,
Albert Einstein incorporated the effect of gravity with special relativity, forming the general
theory of relativity.

The general theory of relativity postulates that the effect of gravity can be characterised as each
gravitational potential source changing the curvature of spacetime. The relationship of
gravitational mass-energy and the shape of spacetime is given by Einstein's field equations:

\begin{equation}
    G_{\mu{}v} + \Lambda{}g_{\mu{}v} = \dfrac{8\pi{}G}{c^4}T_{\mu{}v}
\end{equation}

where \(T_{\mu{}v}\) is the stress-energy tensor \footnote{Tensors are very similar to matrices of
vectors and are typically used to describe mathematical geometric relationships}, \(G_{\mu{}v}\) is
the Einstein tensor, \(g_{\mu{}v}\) is the spacetime metric, \(\Lambda\) is the cosmological
constant, \(G\) is the universal gravitational constant and \(c\) is the speed of light.

An implication of gravity curving spacetime is that massive accelerating objects would cause
`ripples' in fabric of spacetime called gravitational waves. The existence of gravitational waves
remained a theory until 1974, when Russell Hulse and Joseph Taylor discovered a binary pair of
neutron stars that were orbiting each other. After several years of measurement, they found that
the speed at which the stars were orbiting each other was slowing in a manner consistent with the
predictions of the general theory of relativity, showing that gravitational waves did indeed exist
\cite{GravDiscovery}.

There were several experiments performed in the 1960s and 1970s to determine methods to detect
gravitaional waves, resulting in several large laser interferometric detectors (that is, detectors
that use interferometry \textendash{} the phenomena by which waves superpose on each other to create
a resultant wave \textendash{} for detection) being built throughout the early 2000s, including the
American Laser Interferometric Gravitational-Wave Observatory (LIGO) \cite{LIGO}, the Italian Virgo
\cite{virgo} detector, and the German GEO600 detector. The initial observation runs between 2002 and
2011 by the various detectors failed to directly detect any gravitational waves, and as such, the
majority of the detectors began work to increase their sensitivity throughout the 2010s
\cite{aligo}.  The increase in detector sensitivity has brought success in the search for
gravitational waves, with the first direct detection occurring on the 14th of September, 2015
\cite{FirstDetectionPaper}\cite{DetectionWeb}.

Due to their design, the detectors have a significant amount of noise from sources that are not
gravitational waves, in addition to the gravitational waves themselves having very weak signals. As
such, a large amount of data processing needs to be done to the outputs produced by the detectors
in order to filter and extract any possible gravitational waves. These data processors are known as
`\textit{pipelines}', and are mostly created by research groups that are a part of the LIGO
Scientific Collaboration \cite{LSC}. These pipelines are used throughout observation runs for
real-time data analysis.

One such pipeline is the Summed Parallel Infinite Impulse Response (SPIIR) pipeline, created by
Shaun Hooper in 2012 \cite{SPIIRCreate}. The pipeline uses a number of IIR filters \textendash{}
which are commonly used in signal processing for bandpass filtering \textendash{} to approximate
possible gravitational wave signals for detection. The pipeline was further developed by Qi Chu in
2017, by using GPU acceleration techniques to increase the speed of analysis, as well implementing
a method to use a frequentist coherent search \cite{ChuThesis}. The pipeline is currently the
fastest of all existing pipelines, and has participated in every observation run since November
2015, successfully detecting all events that were seen in more than one detector.

The SPIIR pipeline uses GStreamer, a library for composing, filtering and moving around signals, in
addition to the GStreamer LIGO Algorithm Library (\texttt{gstlal}). After receiving data from the
detectors, the pipeline performs data conditioning and data whitening, followed by the usage of the
IIR filters. The data is then combined for post-processing, where events are given sky localization
and then inserted into the LIGO event database \cite{SPIIRGPU2018}.

\subsection{Problem Description \& Goal}
As of \today{}, the SPIIR pipeline supports the use of two or three detectors for gravitational
wave searching \textendash{} the two American LIGO detectors and the Virgo detector. There are
several issues with the current pipeline design that this research project aims to address.

Further detectors are likely to be coming online in the near future, with old detectors occasionally
being removed from detection for maintainance. For example, the Japanese KAGRA detector is
undergoing testing with the goal of being used in the next observation run, and LIGO India is
currently being installed. With the current design of the pipeline, adding and removing detectors is
a significant undertaking that takes a substantial amount of development time.

In addition, if a detector is indeed added to SPIIR, the detector \textit{\textbf{must}} be used for
the coherent post-processing, and can't be used just for synchronization purposes. This presents an
issue as additional detectors are added, as each detector has its own sensitivity, reading
variations and range of observable gravitational wave frequencies, resulting in some detectors being
suitable for searching specific frequency ranges whilst showing no discernable change in output for
other detectors, causing many false negatives.

An ideal architecture for the pipeline would be significantly more composable, able to add and
remove the usage of different detectors for post-processing with minimal effort.
\\

As such, this research project aims to complete a subsection of this idealised architecture. The
project aims to remove the requirement for all detectors to be used for coherent post-processing
with sky localization, and instead aims to provide a generic interface that would allow for any
number of detectors to be used for coherent post-processing, whilst still allowing the unused
detectors to undergo all other parts of the pipeline and remain synchronized with the used
detectors. The project shall explore a number of different possible codebase refactorings as well as
exploring new techniques for efficiently combining \(N\) data sources for coherent search, and shall
measure the performance impact of the changes using a number of benchmarks.

\section{Literature Review}

This research project aims to refactor the SPIIR pipeline codebase and explore techniques for
combining some unknown \(N\) number of data sources for coherent search. Whilst the refactoring part
of the project is somewhat tangential to the aims of the project, looking at the literature for
refactoring will help with the development of the research methodology. The literature for both
refactoring and coherent search are developing and varied.
\\

Refactoring is defined by \cite{Murphy} as the process of changing the structure of software without
changing its behaviour. Of course, this isn't the only way to modify source code to address known
issues. George Fairbanks offers the options of ignoring, refactoring or rewriting code as potential
methods to deal with problems in \cite{Fairbanks}, and makes several distinctions between the
options. According to Fairbanks, the major difference between refactoring and rewriting is the
process by which the code is modified. When refactoring, incremental changes are made to the
odebase, with the major goal being to keep the newly written code integrated with the existing
codebase and tests, as the outputs of a module given some inputs should still remain identical. In
contrast, when rewriting, the new code is written using none of the existing codebase, possibly
resulting in majorly different outputs and possibly even data flow architecture. Unlike with
refactoring, existing tests might not be able to be leveraged, but it becomes much easier to make
sweeping architectural changes to the codebase.

This still leaves the third option \textendash{} ignoring the issues. Fairbanks points out that
ignoring issues simply means that they will have to be dealt with at a later date, and contribute to
`\textit{techinical debt}' \textendash{} a term coined by Ward Cunningham in \cite{Cunningham} to
help explain why otherwise working code may need to be refactored or rewritten, and has since turned
into its own area of academic research, as well as a major focus of industry. Some examples of
activities that ``accrue'' technical debt are; a lack of documentation, implementing sub-optimal
algorithms and a lack of testing. \cite{Allman} notes that technical debt has a number of
similarities to financial debt, in that there can be advantages and disadvantages to accruing the
debt. One such advantage, is that the codebase can be shipped without being entirely complete, and
may indeed reach functional completeness in a shorter time than if the technical debt was not
accrued. Some potential disadvantages, however, include faults in the system, increased maintainance
and extensibility effort, as well as increased time onboarding new members of staff.

Technical debt, then, is clearly an area that needs to be managed over the course of a programming
project, even when rewriting or refactoring. \cite{Fairley} notes that whilst rewriting or reworking
an existing codebase can reduce or eliminate technical debt, the project also risks accumulating
additional debt if not correctly managed. To help with management, \cite{Fairley} suggests adopting
a development process that includes regularly reviewing expected and actual progress, whilst
\cite{Allman} encourages regular internal and external documentation, as well as maintaining an
index of prioritised debts.
\\

The SPIIR pipeline uses a group of IIR (infinite impulse response) filters with time delays to
approximate a matched filter. \cite{SPIIRGPU2018} states that the output of the \(i\)th IIR filter
can be expressed with the equation:

\begin{equation}
    y^i_k = a^i_1y^i_{k-1} + b^i_0x_{k-d_i}
\end{equation}

where \(a^i_1\) and \(b^i_0\) are coefficients, \(k\) is time in a discrete form and \(x_{k-d_i}\)
denotes input with some time delay \(d_i\). After summing the output of the filters, the resulting
signal undergoes coherent post-processing to determine the likelihood of an event having occurred.

Coherent post-processing was introducted in \cite{ChuThesis} as an alternative to coincidence
post-processing. \cite{ChuThesis} states that the multi-detector maximum log likelihood ratio to be
equal to the coherent signal to noise ratio \(\rho{}^2_c\), which can be expressed as:

\begin{equation}
    \rho^2_c = \underset{max\{A_{jk},\theta,t_{c},\alpha,\delta\}}{\ln \mathcal{L}_{NW}}
\end{equation}

where \(A_{jk}\) describes the relative inclination of the detector to the source, \(\theta\) is the
mass of the source, \(\alpha\) and \(\beta\) are the sky directions of the source and
\(\mathcal{L}_{NW}\) is the network log likelihood network.

Whilst SPIIR's coherent search currently only supports the use of two or three detectors,
\cite{ChuThesis} estimates the computational cost of the coherent search to be:

\begin{equation}
    O(2N^3_dN_mN_p)
\end{equation}

where \(N_d\) is the number of detectors, \(N_m\) is the number of sets of IIR filters, and \(N_p\)
is the number of sky locations. \cite{SPIIRGPU2018} discusses a number of optimizations made to the
pipeline, including in sections of the coherent post-processing, but only reduced constant factors
and not the computational cost of the overall process. As such, as more detectors are introduced,
the computational time for coherent search increases cubically.

An interesting parallel can, however, be made to sorting, a very widely studied area of computer
science. Comparison sorts are known to have a lower bound of \(\Omega{}(n\log{n})\) number of
comparison operations \cite{CLRS} which can be increased depending on the algorithmic inputs,
however when implemented on a distributed network, the number of comparators per thread can be
reduced to \(O(\log^2{n})\) using a sorting network, which uses a fixed set of comparisons on the
order of \(O(n\log^2{n})\) \cite{nvidia}. By measuring the total number of comparitors, it would
appear that the distributed algorithm is less efficient, however \cite{nvidia} notes that the
distributed algorithm can sort more keys per second than an optimal \(O(n\log{n})\) algorithm
running on a single thread.

The SPIIR coherent search is distributed on a number of GPUs using CUDA \cite{SPIIRGPU2018}, however
the computational cost in \cite{ChuThesis} is computed as being on the order of the number of
computations, not the number of computation per thread. It is therefore possible that the
computational cost per thread is different to the overall computational cost, and thus as detectors
are added using the existing algorithm, the growth of the computational runtime may not be cubic.

\section{Methods}
\section{Proposed timeline}

\begin{center}
    \begin{longtable}{|p{0.3\textwidth}|p{0.3\textwidth}|p{0.4\textwidth}|} \hline
        \rowcolor[HTML]{454545}
        {\color[HTML]{EFEFEF} \textbf{Task} } & {\color[HTML]{EFEFEF} \textbf{Time} }
        & {\color[HTML]{EFEFEF} \textbf{Description} } \\ \hline
        Proposal Writing (Due 2020-04-20) & 2020-03 to 2020-04 & Write research proposal. \\ \hline
        Literature Review & 2020-04 to 2020-06 & Investigate efficient coherent search methods \\
        \hline
        Analyse Existing Codebase & 2020-04 to 2020-07 & Determine expected inputs and outputs of
        coherent search methods and determine per-thread computational cost \\ \hline
        Oral Progress Report (Due 2020-05-22) & 2020-05 & Give report detailing progress on
        research \\ \hline
        Refactor Existing Codebase & 2020-06 to 2020-08 & Perform changes on pipeline \\ \hline
        Validate New Pipeline & 2020-08 & Compare outputs from changed pipeline to the previous
        version to ensure there is not a regression of results \\ \hline
        Measure Performance Differences & 2020-08 & Determine performance cost of changes \\ \hline
        Abstract & 2020-08 to 2020-09 & Prepare and submit research abstract \\ \hline
        Seminar & 2020-09 to 2020-10 & Prepare and give seminar on research \\ \hline
        Paper Writing & 2020-09 to 2020-10 & Prepare and submit paper on research \\ \hline
    \end{longtable}
\end{center}

\printbibliography[
    heading=bibintoc,
    title={Bibliography}
]{}
\end{document}
