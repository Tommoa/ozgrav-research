\documentclass{article}
\usepackage[utf8]{inputenc}
\usepackage[a4paper]{geometry}
\usepackage[myheadings]{fullpage}
\usepackage[english]{babel}
\usepackage[T1]{fontenc}
\usepackage{sectsty}
\usepackage{appendix}
\usepackage{fancyhdr}
\usepackage{lastpage}
\usepackage{setspace}
\usepackage{amsmath}
\usepackage[yyyymmdd]{datetime}
\usepackage{longtable}
\usepackage[table, xcdraw]{xcolor}
\usepackage[
    backend=bibtex,
    style=numeric,
    sorting=ynt
]{biblatex}
\addbibresource{bibliography.bib}

\renewcommand{\dateseparator}{-}

\onehalfspacing

\pagestyle{fancy}
\fancyhf{}
\setlength\headheight{15pt}
\fancyhead[L]{Student ID: 21963144}
\fancyhead[R]{UWA}
\fancyfoot[R]{Page \thepage{} of \pageref{LastPage}}
\sectionfont{\scshape}

\newcommand{\HRule}[1]{\rule{\linewidth}{#1}}

\title{\normalsize \textsc{GENG555X Research Project}
        \\ [1.5cm]
        \HRule{0.5pt} \\
        \LARGE \textbf{\uppercase{An analysis of the computational complexity of the SPIIR pipeline post-processing}}
        \HRule{2pt} \\ [0.5cm]
        \normalsize \date{\today} \vspace*{3\baselineskip}}

\author{Thomas Hill Almeida (21963144)\\
\\
Supervisor:\\
Professor Linqing Wen
}
\date{}

\begin{document}
\maketitle{}
\tableofcontents{}
\newpage{}

\section{Introduction} \label{sec:intro}

The Summed Parallel Infinite Impulse Response (SPIIR) pipeline, first implemented by Shaun Hooper in 2012 \cite{SPIIRCreate}, uses a number of IIR filters to approximate possible gravitational wave signals for detection followed by post-processing to localize any potential candidates.
The pipeline is currently thought to be the fastest of all existing pipelines, and has participated in every observation run since November 2015, successfully detecting most events that were seen in more than one detector.

Coherent post-processing was introduced in \cite{ChuThesis} by Qi Chu et al as an alternative to coincidence post-processing.
\cite{ChuThesis} states that the multi-detector maximum log likelihood ratio to be equal to the coherent signal to noise ratio \(\rho{}^2_c\), which can be expressed as:

\begin{equation}
    \rho^2_c = \underset{max\{A_{jk},\theta,t_{c},\alpha,\delta\}}{\ln \mathcal{L}_{NW}},
\end{equation}
\\
where \(A_{jk}\) describes the relative inclination of the detector to the source, \(\theta\) is the mass of the source, \(\alpha\) and \(\beta\) are the sky directions of the source and \(\mathcal{L}_{NW}\) is the network log likelihood network.

In \cite{ChuThesis}, the computational cost of the coherent search is estimated to be \(O(2N^3_dN_mN_p)\), where \(N_d\) is the number of detectors, \(N_m\) is the number of sets of IIR filters, and \(N_p\) is the number of potential sky locations.
Further optimizations were made to the pipeline in 2018 \cite{SPIIRGPU2018}, including moving to using GPU acceleration.
Whilst \cite{SPIIRGPU2018} discusses a number of constant time optimizations made to the pipeline, the computational cost of the overall process is not discussed.
In addition, \cite{SPIIRGPU2018} parallelised the pipeline, leading to additional potential changes to the potential overall cost.

This report aims to determine and justify the computational complexity of the existing SPIIR pipeline and provide a framework for any further analysis at a later date.

\section{Background} \label{sec:background}

According to \cite{CLRS}, the theoretical efficiency of a multi-threaded or parallelised algorithm can be measured using the metrics of `span', `work', `speed-up' and `parallelism', all of which should be considered in the context of a directed acyclic graph (DAG) of operations in the algorithm.
The \textit{\textbf{work}} of a parallelised computation is the total time to execute the entire computation sequentially on a single processor, and can be found by summing the total work of every vertex in the DAG.
In comparison, the \textit{\textbf{span}} of a parallelised computation is the maximum time taken to complete any path in the DAG.
It should be noted that the actual running time of a parallelised computation also depends on the number of processors available for computation and how they are allocated to perform different tasks in the DAG, and thus denoting the running time of parallelised computation on \(P\) processors as \(T_P\) is also common practice.
This leads to work being denoted at \(T_1\) (the time taken to run on a single processor) and span being denoted as \(T_\infty\) (the time taken on an infinite number of processors).
Another helpful metric is \textit{\textbf{speed-up}}, which shows how the algorithm scales with additional processors as \(S_P = \dfrac{T_1}{T_P}\).
We can also then define \textit{\textbf{parallelism}} as the maximum possible speed-up on any number of processors, and thus as \(p = \dfrac{T_1}{T_\infty}\).

Using the above definitions, we can re-derive several laws that provide lower bounds on the running time of \(T_P\).

In one step, a computer with \(P\) processors can do \(P\) units of work, and thus in \(T_P\) time can perform \(PT_P\) units of work.
As the total work to be done as per above is \(T_1\), the \textit{\textbf{work law}} states that \cite{CLRS}:

\[
    T_P \geq \dfrac{T_1}{P}.
\]

It is also evident that a computer with \(P\) processors cannot run any faster than a computer with an infinite number of processors, as the computer with an infinite number of processors can emulate a computer with \(P\) processors by using a subset of its processors, leading to the \textit{\textbf{span law}} \cite{CLRS}:

\[
    T_P \geq T_\infty.
\]

It is also useful to use the metrics of `cost' and `efficiency' when analysing parallel algorithms \cite{brent}.
The \textit{\textbf{cost}} of a parallel algorithm is minimised when all of processors are used at every step for useful computation and thus can be defined as \(C_P = P\times{T_P}\).
\textit{\textbf{Efficiency}} is closely related to cost and describes speed-up per processor and can be defined as:

\[
    e_P = \dfrac{S_P}{P} = \dfrac{T_1}{C_P}.
\]

Another helpful theorem for analysis is \textit{\textbf{Brent's Theorem}}, which states that for an algorithm that can run in parallel on \(N\) processors can be executed on \(P < N\) processors in a time of approximately \cite{BrentsLaw}

\[
    T_P \leq T_N + \dfrac{T_1-T_N}{P}.
\]

This can be approximated with the upper bound of \(O(\dfrac{T_1}{P} + T_N)\) \cite{brent}.

Determining the span, work, parallelism, efficiency and cost, and examining the application of Brent's theorem to the computations at hand will allow us to analyse the computational complexity of the SPIIR pipeline.

\section{Analysis} \label{sec:analysis}
\subsection{Maximum reduction with index preservation} \label{sec:reduce}

One of the more common operations in the SPIIR pipeline is the concept of a ``maximum reduction with index preservation''.
Reduction is the idea of taking some array of data and producing a single value from that array, whether it is the total sum of the array or the maximum value of the array and its index in the array as it is in this case.

\cite{reduction} discusses the computational complexity of reduction algorithms in a parallelised context, noting that the best complexity according to Brent's Law is \(O(\dfrac{N}{log N})\) threads each doing \(O(log N)\) sequential work, resulting in a total overall cost of \(O(\dfrac{N}{log N}\times{log N}) = O(N)\).

We can note from our own analysis, that the process of reduction can be parallelised by the use of a binary tree of operations, where each vertex in the binary tree combines the results of the two parent vertices.
In the case of determining the maximum of two numbers, each vertex is identical in the amount of work done, and thus we can determine each vertex to be a unit of work.
As there are \(N\) elements in the original array, we can note that the height of the binary tree is \(log N\), and each level of the binary tree has \(N_l/2\) vertices, thus the total number of vertices in the binary tree is \(\sum_{i=0}^{log N}{2\times{i}} = N\).
Using this information, we can determine that the \textit{\textbf{work}} of a parallelised reduction is \(T_1 = O(N\times1) = O(N)\), and that the \textit{\textbf{span}} of the reduction is \(T_\infty = O(log N \times {1}) = O(log N)\).
Thus, the \textit{\textbf{parallelism}} of the reduction is:

\[
    p = \dfrac{N}{log N}.
\]

Using the span and work laws, we can observe that any algorithm using the above method is bounded by the inequalities \(O(log N) \leq T_P, \dfrac{O(N)}{P} \leq T_P\).
This means that best possible time complexity with \(P\) processors is \(O(log N)\) (the span law).
We can determine the minimum number of processors required to achieve this runtime using the formula \(T_P = O(log N) = \dfrac{O(N)}{P}\), which can be rearranged to
\[
    P = \dfrac{N}{log N},
\]
thus the time complexity cannot improve past \(P = N/log N\) processors.
We can also observe that using \(P = N/log N\) processors gives a \textit{\textbf{cost}} of \(C_P = N\), which is identical to the sequential algorithm.
\\

Functions that include maximum reduction with index preservation will be denoted for clarity with \(M(x)\), where \(x\) is the size of the array being reduced.

\subsection{Determining the number of samples over a signal-to-noise threshold} \label{sec:peaks}

The coherent post-processing in SPIIR determines the number of samples over a signal-to-noise (SNR) threshold in order to not do more work than is required.
The function that is used for determining the number of samples over the threshold (\texttt{peaks\_over\_thresh}) is a sequential algorithm that runs on the CPU, and shall be analysed as such, although there is an alternative GPU-based implementation that is not used.

Initially, the function performs a maximum reduction with index preservation to get the maximum SNR from the IIR filters for each sample.
Recalling from section \ref{sec:reduce} that for maximum reduction with index preservation \(T_1 = O(N)\), and that this operation is performed \(S\) times, where \(S\) is the number of samples, we can determine that this initial reduction has a time complexity of 
\(O(SF)\), where \(F\) is the number of filters.

The function then determines the maximum SNR across the filters found from the previous step by stepping through every combination of samples and removing SNR samples that are using the same filter and have a lower SNR, resulting in a step with a time complexity of \(O(S^2)\).

The function then determines the maximum overall SNR for the input samples (\(O(S)\)) and cycles through every maximum SNR to cluster maximums that are close together to be a single combined maximum.
The number of maximums is bounded by (\(O(\min\{S,F\})\)) as there cannot be more maximums than there are samples or filters.

This gives the overall function a time complexity of \(O(SF + S^2 + S + \min\{S,F\})\), which can be reduced to the dominating terms of:

\[
    O(SF + S^2).
\]

\subsection{Transposing the input matrices} \label{sec:transpose}

The full post-processing function requires that the input matrix is transposed for better memory access such that each row is a different filter, and each column is a different sample.
To transpose the matrix, the GPU function \texttt{transpose\_matrix} is used, thus this should be analysed as a parallel algorithm.

The algorithm in use works by breaking the original array into tiles of size \(32\times32\), and then inserting the transpose of the tile into an output array.
The tiles are further broken down eight processors per row, so each thread does four copies.
We can conceptualise this as a DAG by observing that each tile does not depend on any other tile to be completed, and that each tile is composed of \(32\times8\) interdependent processors, each doing 4 units of work.

Using this observation, we can see that the \textit{\textbf{span}} of the algorithm is \(T_\infty = (32\times8)\times4 = O(1024) = O(1)\), and the \textit{\textbf{work}} is \(T_1 = O(SF)\), where \(S\) is the number of samples and \(F\) is the number of filters.
Thus, the \textit{\textbf{parallelism}} of the transpose is \(p = SF\).

Using the span and work laws, we can observe that the above method is bounded by the inequalities \(O(1) \leq T_P, \dfrac{O(SF)}{P} \leq T_P\).
Thus it can be determined that the best possible time complexity with \(P\) processors is bounded by ratio of available processors to the size of the transposed matrix (the work law).
This gives the function an overall time complexity of:

\[
    O(\dfrac{ST}{P}).
\]

\subsection{Determining the coherent correlation and statistical value of data points} \label{sec:coh_max_and_chisq_versatile}

The scoring metric of different filters and times is determined using coherent correlation and determining their statistical value using a chi squared-based distribution.
These scoring metrics are performed using the GPU function \texttt{ker\_coh\_max\_and\_chisq\_versatile}, and thus should be analysed as a parallelised function.

In CUDA, threads are partitioned three times, into blocks, warps and threads.
Each block is a collection of threads that runs simultaneously, and each warp is a sub-collection of threads within that block, allowing for more coarse-grained execution of algorithms that can be parallelised.

In this function, each block looks at a different SNR maximum (as discussed in section \ref{sec:peaks}) and splits the threads within the blocks for operations on that peak.

\subsubsection{Determining the sky direction of the SNR maximum} \label{sec:skydir}

Initially, each thread within a block looks at a different sky direction and determines the total signal-to-noise ratio (SNR) by summing the SNR of each of the detectors at that given sky direction with the relevant detector arrival time offsets.
The time complexity for the calculation of SNR for a given time offset is \(O(D + D^2)\), where \(D\) is the number of detectors.
The maximum SNR for all the sky directions is then spread across each warp and placed into shared memory before being shared across every thread in the block, which is an application of the parallelised maximum reduction with index preservation function discussed in section \ref{sec:reduce}.
\\

Thus, the \textit{\textbf{span}} of determining the sky direction with the highest signal to noise ratio is \(T_\infty = O(D + D^2 + M_{T_\infty}(S))\) and the \textit{\textbf{work}} is \(T_1 = O(S(D + D^2) + M_{T_1}(S)\), where \(S\) is the number of sky directions and \(M(x)\) is the complexity of the parallelised maximum reduction with index preservation function.
We can further state that the \textit{\textbf{parallelism}} of this is equivalent to the number of sky directions, \(S + S/log S\).

\subsubsection{Calculating signal consistency statistics} \label{sec:stat}

After having determined the sky direction with the highest SNR for a given maximum, the function then calculates a signal-morphology based statistic \(\xi_D^2\) for each detector \(D\).
The statistic is a reduced \(\chi^2\) distribution with \(D\times2 - 4\) degrees of freedom and a mean value of \(1\), and is given in the discrete form by:

\[
    \xi_D^2 = \frac{\sum_{j = -m}^{m} | \varrho_D[j] - \varrho_D[0] A_D[j]|^2}{\sum_{j = -m}^{m} (2 - 2|A_D[j]|^2)},
\]
where \(\varrho\) is the coherent SNR, \(A_D\) is the vector of the correlation of the given filter with the output from the detector and \(2\times{m}\) is the number of samples.

The numerator of the statistic is calculated by splitting the number of samples between the threads of a block, followed by combining the results of the statistic across each warp and then each block.
The combination of the statistic across each warp and block is a modification of the parallelised maximum reduction with index preservation discussed in section \ref{sec:reduce} that uses addition instead of maximum as the combining binary function.
Thus the \textit{\textbf{span}} of calculating the statistic is \(T_\infty = O(D\times{M_{T_\infty}(N)})\) and its \textit{\textbf{work}} is \(T_1 = O(D\times{M_{T_1}(N)})\), where \(N\) is the number of samples.
We can then state that the \textit{\textbf{parallelism}} of calculating the statistic is equivalent to the parallelism of the reduction, \(O(N/logN)\).

\subsubsection{Generating time-shifted background noise statistics} \label{sec:background}

% Why do we regenerate the background noise for every peak? Other than where we output
% the background noise, there's no difference between any of the peaks
%
% Removing background noise generation from this function would increase the speed of
% this function by an order of magnitude
The function then performs a number of time shifts on background noise for use with the significance estimation.
The generation of a single background statistical variant is equal to the total work of the function so far, save that instead of using blocks for every peak, each warp looks at a different time shift.
Thus, whilst the theoretical time complexity does not change, the number of processors available is smaller, so the actual runtime each loop is approximately the warp size slower.

\subsubsection{Overall computational cost} \label{sec:versatile_cost}

Overall, this function has a \textit{\textbf{span}} of \(T_\infty = 2(D + D^2 + M_{T_\infty}(S) + DM_{T_\infty}(N))\), and has \(T_1 = P(S(D + D^2) + M_{T_1}(S) + DM_{T_1}(N) + B(S(D + D^2) + M_{T_1}(S)))\) \textit{\textbf{work}}, where \(P\) is the number of SNR maximums and \(B\) is the number of times shifts made to background noise.

\subsection{Calculating heat skymaps} \label{sec:coh_skymap}

If the coherent SNR exceeds a threshold, the post-processing produces a skymap of the highest SNR in the GPU function \texttt{ker\_coh\_skymap}.

The function determines the highest maximum SNR by using the maximum reduction with index preservation technique discussed in section \ref{sec:reduce}.
Following this, the function re-performs the process discussed in section \ref{sec:skydir} with additional sky directions and without the reduction to generate the final skymap.

As such, this function has a \textit{\textbf{span}} of \(T_\infty = M_{T_\infty}(P) + D + D^2\) and total \textit{\textbf{work}} of \(T_1 = M_{T_1}(P) + S(D + D^2)\).

\printbibliography[
    heading=bibintoc,
    title={Bibliography}
]{}
\end{document}
